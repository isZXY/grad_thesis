% !TeX root = ../thuthesis-example.tex

\begin{acknowledgements}
四季流转,转眼间就来到了我硕士生涯的尾声。回顾过去,有许多记忆犹新的片段依然历历在目,它们就像拼图一样构成了完整的三年。我感到幸运,因为一路上有许多无声支持与陪伴,伴我走过或是快乐的,或是具有压力的日子。因此,我想在此致谢:

衷心感谢我的导师王智副教授,感谢您在整个研究生的过程中提供的悉心指导与鼓励。在三年多的时间里,您深刻并富有远见的见解、专业的行事和深厚的累积就像是在科技的浪潮和风暴中的一座灯塔,让我看到应该探索的朝向。您不仅给予了学术上的入门与启发,还提供了许多与真实工业与学术界接触的宝贵机会,在此期间学会了与不同业界的人展开合作和沟通。更重要的是,在您的言传身教下,我学会了严谨与高效的做事态度与方法,以及良好的面对挑战的心态。这些习惯和心态将会伴随我受益终身。

感谢现在于腾讯进行技术研究的吴成磊大师兄和字节跳动的张伟技术专家,你们的指导和经验引领我在相关的专业领域上有更深刻的认识,实践的技术上更加精通。

感谢MMLab实验室中的所有成员,感谢24级博士吴铎为我带来了新的灵感与启发,并给予详细的指导。感谢20级硕士(24级博士)解书照,给予我一直以来默默的帮助。感谢21级硕士代诗绮,你的努力和行事方式一直激励着我,科研的思路也让我有所启发。感谢21级硕士(24级博士)路荣伟提供的科研学习的机会。感谢LM小组的王瀞禾、孙乐、王智民、张晏宁,我们共同学习和进步。我还想感谢22级同级的同学刘文洋、邹岷强、闫柏旭、卢美子,23级姜佳成在研一研二期间经常的鼓励和一同游玩解压的快乐时光。感谢23级博士李也、22级硕士李迎欣在Retreat会议中结下的深厚友谊和难忘时光。感谢22级同级同学杜南洋、张伟翔、姚为、张岑岳、张露丹、方子介,我们共同努力,互相帮助面对各种挑战。感谢23级师弟葛士嘉、张瑾睿、宋佳俊、蒋沁廷、李群、李孝杰曾一起游玩放松。24年,又有许多新生加入,我们共同参与了MMLab年会,在此中认识了充满活力的新人,共同度过了意犹未尽和不舍离开的时光。

感谢我的父母,你们是我最坚实和实在的后盾,在生活上无微不至的关心和对我无私的付出令我感动,这让我能够毫无后顾之忧地不断追寻自己的梦想。感谢我的女朋友,这么多年的异地你不曾放弃,在许多个困难的瞬间都是你重新给予我力量。

我还想感谢清华给予的广阔平台和优渥资源。清华所拥有的凝聚力让我受到许多受益匪浅的帮助;其校训精神和价值观也深刻地感化并滋润着我们,激励我们不负社会,警记优渥的环境下所塑造的能力优势的意义是用于帮助有需要的人。

最后,我想感谢自己的努力和梦想。在一个努力既有回报的时代下,相信每个人都会或早或晚得到奖赏。但即使有一天,回报不再显著,我们会庆幸和真诚地感谢那些黄金岁月。


\end{acknowledgements}

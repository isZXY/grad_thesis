% % % !TeX root = ../thuthesis-example.tex

% \chapter{补充内容}


% \section{大规模数据测量中的数据属性及筛选方法}\label{shaixuan}


% 被测量的游戏及每款游戏使用的数据集如表\ref{table_flow_count}所示。数据有两种不同粒度类型:流粒度持续时间是指从用户发起连接请求开始,直到用户关闭连接为止,形成一组统计数据为一次统计,作为单条数据进行使用的数据;而帧粒度持续时间则是指单个流中每一帧画面显示为一次统计,一个流形成一张表格。后者的粒度比前者更细,可以获得更多细节统计信息,收集用于进一步探索,两者共同使用形成测量结果。
% \begin{table}[ht]
\centering
\caption{用于统计游戏的流属性统计}
\begin{tabular}{@{}ccccc@{}}
\toprule
\textbf{游戏}        & \textbf{流粒度持续时间(小时)} & \textbf{帧粒度持续时间(小时)} \\ \midrule
2D-RPG I             & 3707.74                  & 910.23                    \\
2D-RPG III           & 2558.73                  & 460.17                    \\
3D-RPG I             & 889.68                   & 49.83                     \\
动作游戏 I           & 1194.60                  & 102.73                    \\
A-RPG I              & 529.15                   & 87.37                     \\
休闲游戏 I           & 70.47                    & 3.29                      \\
CCG I                & 200.42                   & 12.80                     \\
FPS I                & 1360.89                  & 277.21                    \\
FPS II               & 151.32                   & 79.51                     \\
FPS III              & 570.81                   & 62.75                     \\
FPS IV               & 281.95                   & 20.71                     \\
MMORPG I             & 638.79                   & 82.72                     \\
MOBA I               & 685.65                   & 378.66                    \\
MOBA II              & 247.16                   & 98.07                     \\
运动游戏 I           & 1068.06                  & 249.90                    \\
运动游戏 II          & 155.76                   & 35.31                     \\
运动游戏 III         & 1376.29                  & 0.13                      \\
TPS I                & 331.30                   & 25.43                     \\ \bottomrule
\end{tabular}
\label{table_flow_count}
\end{table}


% 从大规模测量中获得的原始数据,由于记录中存在异常使用数据,往往偏离理想数据。为了确保数据测量的准确性,需要应用某些规则,尽可能排除由于用户或设备链路问题导致的异常数据。可能的异常数据情况包括:


% \begin{itemize} 
% \item \textbf{用户提前退出:}该情况指的是用户与服务器建立连接后,但未实际进行游戏便退出。此类连接通常持续时间较短,不超过 2 分钟。 

% \item \textbf{僵尸流量:}该情况指的是客户端与服务器的连接异常断开,但流量未停止并继续计数,导致数据不正确。这样可能导致流量被计量几个小时甚至超过一天,超过了正常可接受的长时间使用时长。 

% \item \textbf{频繁切换网络:}为了排除用户网络连接不稳定的情况,我们选择常见的网络类型(2.4GHz 和 5GHz Wi-Fi、以太网或蜂窝数据)。在单次连接中,频繁切换网络链接的情况将不被纳入统计。 

% \item \textbf{极低的实际编码比特率:}实际编码比特率低于设置比特率的十分之一,通常是由于长时间静态或极小的画面,导致需要传输的残余信息非常少,从而导致极低的比特率。这表明尽管用户已连接,但并未进行任何操作。 

% \item \textbf{低交互频率}:用户客户端接收到的输入频率极低,如键盘、鼠标或游戏手柄的输入频率。此类情况也表明用户已建立连接,但未进行任何游戏活动。 

% \item \textbf{异常的客户端处理时间:}通常,在性能稳定的机器上接收和解码视频不应超过某个时间限制。处理时间波动较大或连接时间过长的情况也应排除。在处理延迟过高的情况下,用户无法正常游戏,而连接时间过长则表明客户端存在异常行为,因此需要排除。 


% \end{itemize}
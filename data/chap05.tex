% !TeX root = ../thuthesis-example.tex

\chapter{总结与展望}
\section{研究总结}
在当下的科研与多数互联网通信服务中,如何在复杂网络环境中提升传输效率以及保证服务良好的用户体验是重要的研究问题。合理的传输策略能够有效增加信息传递效率,提升用户对服务的满意程度,降低通信开销和成本,因此设计更高效的传输策略对于内容服务提供商、网络通信运营商和用户受众都发挥着举足轻重的作用,尤其是在诸如网络拥塞控制、自适应视频码率、实时通信系统发送速率控制等任务上具有非常可观的优化收益。本文的研究旨在结合最新的人工智能决策模型来优化这类任务中的传输策略,通过精细化具体传输内的网络场景和用户偏好展开策略和工作范式的优化。

本文通过从应用场景需求异质性和环境性能适配差异性两个角度展开研究。对于应用场景需求异质性,研究选定了云游戏下实时传输速率控制场景,对算法展开优化。本文认为,不同的场景中延迟对用户满意度的影响存在差异,因此通过测量获得具体的“敏感度”。研究首先针对云游戏中不同的游戏场景展开了大规模的时延与使用时长关系的数据测量并进行了时延对应的用户满意度主观打分实验,获得了场景下真实的用户反馈。为将上述的用户反馈数据应用至算法设计中,本工作使用最小二乘估计将每个场景下的数据映射为连续的敏感性函数,获得了多条不同的敏感性曲线。在实时传输速率控制上使用强化学习的框架完成交互,并将前获得敏感性曲线应用在强化学习的奖励函数,获得匹配场景实际需求的控制策略。

对于环境性能适配差异性,本文通过构建一个工作在策略之上的范式,避免网络控制策略陷入到局部最优而非全局最优解中,进而增强差异化的网络环境下的传输整体性能。研究首先通过更新网络任务与策略规划系统的策略选择和交互机制增加了切换机制,然后构建策略规划系统。策略规划系统利用预训练模型强大的规划能力作判断,并采用了序列化的决策模型适配模型的嵌入层。为了进一步对齐模型的规划输入输出,使用预采集的数据完成了离线强化学习模型权重的调优,并在自适应码率上展开了案例研究,为多种网络任务提供了一个新的网络规划范式。

综上所述,本文借助人工智能决策的框架和模型完成了一个云游戏场景下的传输策略优化和一个网络控制策略的选择范式,能够针对于异质化的网络场景提供差异化的码率决策,提升了网络利用率;同时网络控制策略的选择范式能够避免单一的算法在变动的网络环境中因长期工作而陷入次优解。本文提出的两个模块独立工作,能够在整个传输过程中和多种应用中提供更优异的性能。

\section{未来展望}
本文针对于异质网络场景下的网络传输任务展开了较为深入的探究。随着该问题引起越来越多的关注,该领域仍然存在较大的研究空间,包括以下两个改进方向:

(1) 场景下弱用户参考基准的时延敏感度的获取及泛化

尽管使用敏感度曲线细分的场景能够有效地提高传输效率、增加用户体验以及降低成本,当前工作对于时延敏感度评估的过程仍需高度依赖人类主观偏好的反馈,而开展大量此类的时延和测量是极大的成本开销,且测量的结果只能适用于极小一部分场景,例如本研究中时间和空间变换度值接近的场景。因此,将时延敏感度进一步与人类直接的打分解耦合能够加强场景化算法的应用,这可以通过多种方式实现:使用具有强代表性的无参考指标,例如Deep-BVQM \cite{jamshidi2022deep} 使用无参考指标PSNR的方式实现了视频帧级别的质量预测;或是借助诸如大模型智能体(LLM Agent)的新人工智能技术,通过少量采集数据即可构建敏感度映射关系,并将相关数据映射到更多差异化场景下。

(2) 网络规划范式规划结果的可解释性与配置

本文现阶段设计的网络规划范式决策模型能够作出策略的规划,但仍然存在进一步优化的空间。针对于大多网络要求的即时性和简明性,现有的工作范式未能阐明其规划和决策的原因。而大语言模型本身具有强大的自然语言能力,便于以人类易于理解的方式展开阐述。如何利用该种能力对网络规划范式规划结果进行解释,乃至进一步对选定策略的配置项展开配置是重要的未来研究方向,它将透明化大模型决策过程并进一步优化配置性能,提升配置能力。透明化的另一个好处是便于模型的蒸馏,降低后续的长期部署成本。